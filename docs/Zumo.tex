\documentclass{article}

\usepackage[a4paper,margin=2.5cm]{geometry}
\parindent=0pt
\frenchspacing

\usepackage[none]{hyphenat}
\usepackage{parskip}
\usepackage[hyphens]{url}
\usepackage{hyperref}
\usepackage{booktabs}

\hypersetup{colorlinks, citecolor=red, filecolor=red, linkcolor=black, urlcolor=blue}

\begin{document}

\title{Setting up the Zumo robot}
\author{Leon Helwerda}
\date{\today}

\maketitle

\section{Introduction}

The Zumo has documentation that does require some searching in case one wants 
to connect it differently than one usually would (with only an Arduino on the 
Zumo Shield). Since we might want to do certain processing on the Raspberry Pi, 
or maybe even remove the Arduino out of the equation, we need to know which 
pins are used for what specifically and forward information to the Pi. The Zumo 
has a software library, written in C and Arduino code, that offer more details 
about the pin uses. Additionally, documentation and data sheets corresponding 
the individual parts also provide more information.

\section{Pin numbering}
The following pin numbers are useful to know for the Zumo Shield robot. We will 
have to make appropriate connections from an Arduino (which corresponds nicely 
with the shield) or a Raspberry Pi (which then has its own numbering). One can 
also solder an additional array of pins on the shield to forward signals from 
the Arduino to the Pi.

There exists a difference between analog and digital pins in numbering, but as 
far as it seems both can be used for normal purposes.

The following pins are used for various components and interfaces with the Zumo 
Shield and Arduino Uno:

\begin{itemize}
  \item LEDs: two red/blue power LEDS with no pin control, one yellow LED 
        controllable using pin 13.
  \item Push buttons: a reset button that is connected to the RST pin. Usually 
        this is connected to a RESET pin on the Arduino. Note that we can add 
        other peripherals to the RST pin to cause the Arduino to reset, or 
        connect it to a DTR line of a serial connection to reset it from the 
        Raspberry Pi. Be sure that there is not too much power on this line and 
        do not power it when the Zumo itself is not powered, otherwise the 
        brown-out detection will kick in and cause a reset loop. \\
        Aside from the reset button, there is a user button that is connected 
        to pin 12.
  \item Motor: pins 7 (left) and 8 (right) are for direction. A low signal 
        causes a motor to drive forward, and a high signal drives it backward.  
        Pins and pins 9 and 10 are for motor speed. The speeds are sent as PWM 
        values that are between 0 (no throttle) and 255 (maximum speed). The 
        Zumo libraries use 0 to 400 and convert that using a multiplier 
        $51/80$, unless 20KHz PWM is used.
  \item Buzzer: can be used to play simple sounds by passing PWM values. Can be 
        controlled using pin 3 or pin 6, depending on the Arduino being used. 
        By default, the buzzer is disconnected, but it can be connected by 
        replacing the blue shorting block from its default horizontal position 
        to the one marked as 3/328P (Uno) or the one marked with 6/32U4 
        (Leonardo), respectively. The Zumo buzzer library can be used to play 
        more complicated songs.
  \item Battery monitor: One can monitor the battery voltage of the Zumo shield 
        using the A1 analog pin, but only if the jumper in the middle of the 
        board is connected by soldering a line between the A1 and Battery Level 
        holes. Otherwise, one can use A1 for other purposes.
  \item Inertial sensors: a combined accelerometer and magnetometer (compass) 
        as well as a gyroscope are connected via I2C. By default, these are 
        disconnected, but they can be forwarded to the SDA and SCL pins on the 
        bottom right of the board, close to LED13. Arduino versions older than 
        Uno R3 do not have these pins, which could allow us to connect them to 
        the Raspberry Pi, for example to achieve another external sensor which 
        could work with EKF2. On the Arduino Uno R3, the SDA and SCL pins are 
        duplicated to analog pins A4 and A5, respectively.
  \item I2C: pins 2 and 3 (Leonardo), or analog pins A4 and A5 (Uno), for 
        forwarding SDA and SCL respectively of a mounted Arduino. These pins 
        can have different purposes, including a connection with the 
        compass/gyroscope or a control line to the reflectance sensor array. 
        The sensor array can use a shorting block to determine which one to 
        use, which means it uses the pin that is not already in use for I2C.
  \item RX/TX: pins 0 and 1 for forwarding serial communication with a mounted 
        Arduino. One can also use any free pin with SoftwareSerial to use 
        a different UART RX/TX connection. For example, when using an Uno R3 
        with the battery level jumper disconnected, pins 6 and A1 can be freely 
        used for RX/TX.
  \item Reflectance sensor array: Six IR sensors and additional status LEDs. 
        Depending on how a shorting block is placed on top right of the sensor 
        array, one can enable or disable the sensors using a pin. If the block 
        is near to the Zumo shield then pin 2 is connected to LEDON, if it is 
        away from the shield with a bare pin visible, then pin A4 is connected 
        to LEDON. Send a high signal to the correct pin to enable the sensor 
        array, and a low signal to disable again; the visible red status LEDs 
        should show this. \\
        The IR sensors themselves are read from left to right, when facing the 
        forward driving direction with the shorting block in the back of the 
        array, as pins 4, A3, 11, A0, A2, and 5. \\
        All pins are forwarded via the front expansion board to the pins with 
        extenders placed on the Arduino mount. The sensor pins are both input 
        and output pins. In order to start a measurement, send a HIGH to all 
        pins and set them to be output pins for about 10 microseconds, then 
        make them input pins and set to LOW to disable pull-up resistors. The 
        pins will then send each send a pulse whose length determines the 
        blackness of the surface. A zero length pulse means completely white; 
        a pulse approaching infinite length (always 1) means completely black. 
        One should use a timeout of e.g. 2000 microseconds and consider all 
        pulse lengths above a certain threshold (e.g. 300 microseconds) to be 
        black.
\end{itemize}

\section{Connecting to Raspberry Pi}

This section describes our setup for communicating between the Raspberry Pi and 
the Arduino mounted on the Zumo robot, as well as mounting everything on top of 
each other. Additionally, there are some notes and warnings about the use of 
the whole package to keep it safe and sound.

Before connecting anything, ensure that the Arduino has the most recent 
grid-following Arduino code uploaded to it. If it does not, then we need to 
upload the code to the device. Ensure no power or other serial devices (such as 
the USB-TTL programmer) are connected to it, connect a type B USB cable between 
it and a laptop, start the Arduino IDE, open the {\tt zumo\_grid.ino} file and 
upload it. After waiting for the upload to finish (indicated by flashing RX/TX 
LEDs) and verifying that the program uploaded correctly (indicated by the 
status LED and a buzzer sound), disconnect the cable again.

To assemble the robot, we need to use the exterior pins of the Zumo shield for 
connecting a USB-TTL programmer for serial communication. Thus, if the Zumo is 
assembled, we need to disassemble in order to solder additional pins. Follow 
these steps to disassemble it safely, and after soldering the necessary 
exterior pins, reverse the order to assemble it again: remove the caterpillar 
tracks around the wheels, remove the battery cover and any batteries in the 
bottom of the chassis, remove the screws holding the front plow and the back of 
the shield, squeeze the spring of the battery holder that is closest to the 
{\tt -} sign and push it through a small hole (but larger than the other holes 
above the battery pack), remove the insulation plates between the shield and 
the chassis as well as the chassis itself, and finally bend the motors away 
without breaking the motor leads to allow access to the pin soldering points.

We assume that we have a battery case that can be laser-cut from a model 
provided in the {\tt casing} directory. The individual plates of the case can 
be connected using superglue; it is useful to keep the top cover of the case 
loose such that it can be removed as if it were a lid, in order to replace 
batteries. First put spacers and/or screws between the holes of the Arduino Uno 
R3 and the bottom plate of the casing.

Next, connect some cables to the following exterior pins of the Zumo so that 
we do not need to connect them when they are not as reachable anymore. Keep 
track of their names or colors to tell them apart. The RX and TX pins may 
differ due to the use of software serial; the {\tt zumo\_grid.ino} program 
contains the correct pins to use. Table~\ref{tab:serial} shows an example 
connection scheme. Do not connect the VCC or CTR pins of the USB-TTL; you can 
shield them with a loose cable if this is safer.

\begin{table}[h!]
  \centering
  \begin{tabular}{rcrl}
    \toprule
    USB-TTL pin & Example wire color & Arduino pin & Arduino pin function \\
    \midrule
    DTR & Green & RST & Reset \\
    TXD & Yellow & A5 & softSerial RX \\
    RXD & Blue & A1 & softSerial TX \\
    GND & Black & any GND & Ground \\
    \bottomrule
  \end{tabular}
  \caption{Connection setup for the USB-TTL programmer and Arduino.}
  \label{tab:serial}
\end{table}

Put the battery pack in the casing, and attach the BattBorg to the vertical 
plate with two holes; one for the power cables and one for a spacer/screw. 
Ensure that they cannot short circuit or make current flow to circuits that are 
already powered by another power source. Then securely mount the Arduino on the 
Zumo so that all the pins align with the middle rows of the Zumo pins. Note 
that the size of the case as well as the Zumo caterpillar tracks may make it 
difficult to reach the exterior pins, so one may have to dismount the Arduino 
entirely if one would need to disconnect these for reprogramming.

Fix the (bare) Raspberry Pi on the top cover of the casing using screws. ThePi 
case excluding the bottom cover can be put on top of the Pi. Connect the 
infrared sensor as explained in the documentation for that component; use GPIO 
pin 9 instead of 6 to be able to connect the Battborg. Other peripherals should 
be connect as well. Connect the three-pin M/F cable to the Pi, with the black 
cable on pin 6. Do not connect this cable to the BattBorg just yet.

Insert the Arduino USB-TTL programmer into the bottom right USB port (if not 
using XBees, any port can be used subject to cable length). Connect any other 
peripherals such as the XBee (in the top right USB port), and recheck whether 
everything is connected correctly.

Finally, ensure the Zumo has batteries and turn it on. Then connect the 
BattBorg cable to the BattBorg, with the black cable closest to the black/red 
battery cable connectors in the outermost triple of the female connector block. 
Always perform this action in this order for reasons described below.

Note that the Arduino {\tt zumo\_grid} code, when it is connected to the Zumo 
and the DTR line is not resetting the Arduino, plays a startup buzzer sound, 
and then waits for the serial connection to be set up completely. Thus, the 
robot waits for the Raspberry Pi to start and set up before it performs the 
calibration, plays a success sound, and waits for other movement commands. Once 
the Pi is booted, it enables the USB interfaces. This by defaults sends a high 
signal to the DTR line, causing the Arduino, but not the Zumo, to turn off 
(indicated by the LED going off on pin 13, but not the power LEDs). Only once 
the vehicle is set up within the mission script will it send a low DTR signal, 
turning the Arduino on again. Depending on the mission, the Arduino will then 
immediately calibrate, moving around. This differs between missions, and for 
the RF sensor mission, it depends on whether a mission dump file exists on the 
Pi. Otherwise, this mission waits until all control panel waypoints have been 
sent before arming. Either way, there should be enough time after powering to 
place the robot at its starting location. Of course, the mission itself only 
starts when the infrared sensor receives a start button press.

Another fair warning is that the DTR/RST line might exhibit weird behavior. 
When the Zumo power is turned off while the Pi sends a low DTR signal, the LED 
on pin 13 sometimes still receives enough current via RST to be on. When you 
turn on the Pi before the Zumo (in contrast to the precautions in these steps) 
then the current on the RST causes the brown-out detection of the Arduino to 
kick in continuously, causing the LED to flash and the Zumo to make a clicking 
sound. Depending on the Arduino, this may happen in different intervals. Both 
scenarios should not harm the devices any more than connecting the DTR/RST in 
the first place, but it is preferable to avoid these cases. Another good rule 
here is to always power off the Raspberry Pi first before one turns off the 
Zumo. In case of emergency, turning off the Zumo motors may be necessary and 
fine, but the infrared sensor's stop button, which resets the Arduino, was 
tested to work just as well.

On that note, the stop button (or some other exception in the Python code) 
resets the Arduino by stopping the Raspberry Pi service and restarting it. In 
between the restart, the Arduino plays the startup sound because it is no 
longer receiving a signal to the DTR/RST line. However, it will not yet 
calibrate, since this only happens after creating the serial connection and 
resetting one more time. Thus, you have enough time to reposition the Zumo 
before it wants to calibrate (for missions that do not delay the arming step).

Some settings overrides on the Raspberry Pi are hardcoded in {\tt 
raspberry-pi-start.sh}. Other settings can be altered via the settings view in 
the control panel. To update code, we use the Ethernet connection to SSH to the 
Pi and use {\tt git pull} or other Git commands, make local changes to 
settings, etc. This can be done while everything is connected, but make sure 
the Zumo does not start to calibrate during this by either keeping a mission 
that awaits a start button press before running, letting it stay at an ended 
mission, or halting the service using {\tt make stop} (disallowing it to 
restart as well). As stated at the start of this section, for Arduino code 
updates, all the cables must be disconnected and the Zumo must be powered off 
before connecting the USB type B cable.

\section{References}

\begin{itemize}
  \item Zumo pin numbering: \url{https://www.pololu.com/docs/0J57/5}
  \item Arduino Uno pinout diagram: \url{http://i.stack.imgur.com/EsQpm.png}
  \item Arduino internal numbering details: 
        \url{https://github.com/arduino/Arduino/blob/master/hardware/arduino/avr/variants/standard/pins_arduino.h#L50}:
  \item Labeled image of the Zumo shield pins: 
        \url{https://www.pololu.com/picture/view/0J5572}
  \item How to solder the battery level monitor jumper and how to read the 
        battery voltage of the Zumo: 
        \url{https://www.pololu.com/docs/0J57/3.c}
  \item Reflectance sensor array pin numbering: 
        \url{https://github.com/pololu/zumo-shield/blob/master/ZumoReflectanceSensorArray/ZumoReflectanceSensorArray.h#L152}
  \item Documentation of the reflectance sensor array: 
        \url{https://www.pololu.com/docs/0J57/2.c}
  \item Internal reflectance sensor implementation: 
        \url{https://github.com/pololu/zumo-shield/blob/master/QTRSensors/QTRSensors.cpp}
  \item Zumo schematic: 
        \url{https://www.pololu.com/file/download/zumo-shield-v1_2-schematic.pdf?file_id=0J779}
  \item Arduino Uno schematic: 
        \url{https://www.arduino.cc/en/uploads/Main/Arduino_Uno_Rev3-schematic.pdf}
  \item Arduino Uno: \url{https://www.arduino.cc/en/Main/ArduinoBoardUno}
  \item Assembling the Zumo Shield and chassis: 
        \url{https://www.pololu.com/docs/0J57/2.b}
  \item Connecting the BattBorg: \url{https://www.piborg.org/battborg/install}
\end{itemize}

\end{document}
